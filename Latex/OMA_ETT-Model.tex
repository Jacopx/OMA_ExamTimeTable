% Front Page
\title{Exam Time Table\\ Model}
\author{\textbf{Group 3} [MA-ZZ] \bigskip\\
        Jacopo Maggio\\Stefano Munna\\Jacopo Nasi\\Andrea Santu\\Marco Torlaschi}
\date{\bigskip\bigskip\today}

\documentclass[12pt]{article}
\usepackage[utf8]{inputenc}
\usepackage[italian]{babel}
\usepackage{geometry}
\usepackage{indentfirst}
\usepackage{mathtools}
\usepackage[usenames, dvipsnames]{color}
\usepackage{float}
\usepackage{amssymb}
\usepackage{ifsym}
% Measurement
\geometry{ a4paper, total={170mm,257mm},left=35mm, right=35mm, top=35mm, bottom=35mm }

\begin{document}

\begin{figure}
  \centering
  \includegraphics[width=10cm]{images/polito.pdf}
\end{figure}

\maketitle % Print credits
\newpage

% Model pages
\section{Mathematical Model}
This document contains the mathematical model designed for the assignment of Optimization Methods and Algorithms (2017/2018) for the GROUP 3 of the second part of the course [MA-ZZ].

\subsection{Variables}
We have defined two main boolean variables ($x,y \in \{0,1\} $). The first is related to the exam in time slots:
\begin{equation}
  \begin{gathered}
    x_{e,t} = \begin{cases} 1 \rightarrow $ \textit{If the exam \textbf{e} is assigned to timeslot \textbf{t}}$ \\ 0 \rightarrow Otherwise \end{cases}
  \end{gathered}
\end{equation}
The second one is used for student and exam enrollment:
\begin{equation}
  \begin{gathered}
    y_{s,e} = \begin{cases} 1 \rightarrow $ \textit{If the student \textbf{s} is enrolled in exam \textbf{e}}$ \\ 0 \rightarrow Otherwise \end{cases}
  \end{gathered}
\end{equation}
We have also defined the number of students enrolled in both conflicting exams e and e' with the variable:
\begin{equation}
  \begin{gathered}
    n_{e,e'} = \sum_{s = 1}^{S} y_{s,e} \cdot y_{s,e'}\\
    \forall e, e' \in \{1,...,E\} \wedge (e \neq e')\\
    s\in\{1,...,S\}\\
  \end{gathered}
\end{equation}
Parameters defined by the assignement:
\begin{itemize}
  \item S: Number of students $\rightarrow S \in \mathbb{N}$
  \item $t_{max}$: Number of time slots $\rightarrow t_{max} \in \mathbb{N}$
  \item E: Number of exam $\rightarrow e \in \mathbb{N}$
\end{itemize}

\subsection{Objective Function}
The pourpose of the model is to MINIMIZE the following expression:
\begin{equation}
  \begin{gathered}
    \sum_{e=1}^{E-1}\sum_{e'=e+1}^{E}\sum_{t=1}^{t_{max}-5}\sum_{t'=t}^{t+5} (\frac{2^{5-i}}{|S|} \cdot n_{e,e'} \cdot x_{e,t} \cdot x_{e',t'})\\
    i = t'-t
    \label{eq:fir}
  \end{gathered}
\end{equation}
Since this equation is not a linear function (due to the product of two decision variables), we need to introduce a new variable to linearize it:
\begin{equation}
  \begin{gathered}
    K_{e,t,e',t'} = x_{e,t} \cdot x_{e',t'}\\
    K_{e,t,e',t'} \in \{0,1\}
    \label{eq:sec}
  \end{gathered}
\end{equation}
and to define some constraints(\ref{eq:linearize}) that guarantee the correct transition from the non-linear function to its linearized form.

\subsection{Constraints}
Two  are the main constraints that must be defined to guarantee the feasibility of the solution. 
The first is the one related to the possibility to schedule an exam only once during the examination period:
\begin{equation}
  \begin{gathered}
     \sum_{t=1}^{t_{max}} x_{e,t} = 1\\
     \forall e, e' \in \{1,...,E\}
  \end{gathered}
\end{equation}
The second one is to prevent two conflicting exams being scheduled in the same time slots:
\begin{equation}
  \begin{gathered}
     n_{e,e'} \leq |S| \cdot w_{e,e'}\\
     K_{e,t,e',t'} \leq 1 - w_{e,e'}\\
     w_{e,e'} \in \{0,1\}
  \end{gathered}
\end{equation}
The two inequalities ensure that \textit{K_{e,t,e',t'}} will be zero if \textit{n_{e,e'}} is bigger than zero (and consequently exams must be scheduled in different time slot). 
The following constraints are defined, as already said, for \textbf{linearization of the objective function}:
\begin{equation}
  \begin{gathered}
     K_{e,t,e',t'} \leq x_{e,t}\\
     K_{e,t,e',t'} \leq x_{e',t'}\\
     K_{e,t,e',t'} \geq x_{e,t} + x_{e',t'} - 1\\
     \label{eq:linearize}
  \end{gathered}
\end{equation}
The first two inequalities ensure that \textit{K_{e,t,e',t'}} will be zero if either \textit{x_{e,t}} or \textit{x_{e',t'}} are zero. The last inequality will make sure that \textit{K_{e,t,e',t'}} will take value 1 if both binary variables are set to 1.

\subsection{Final Objective Function}
The final version of the objective function is obtained merging the equation \ref{eq:fir} and \ref{eq:sec} with the following result:
\begin{equation}
  \begin{gathered}
    min\sum_{e=1}^{E-1}\sum_{e'=e+1}^{E}\sum_{t=1}^{t_{max}-5}\sum_{t'=t}^{t+5} (\frac{2^{5-i}}{|S|} \cdot n_{e,e'} \cdot K_{e,t,e',t'})
    \label{eq:final}
  \end{gathered}
\end{equation}

\end{document}
